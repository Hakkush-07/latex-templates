\section{Introduction}

Curves have been interesting for a long time. Intuitively, a curve may be thought of as the trace left by a moving point. The first definition of a curve in the literature of mathematics appeared in Euclid's Elements.

\begin{quote}
    The [curved] line is the first species of quantity, which has only one dimension, namely length, without any width nor depth, and is nothing else than the flow or run of the point which will leave from its imaginary moving some vestige in length, exempt of any width.
\end{quote}

In modern mathematics, curves have various definitions depending on the settings they are in. In this setting, the main actor is the algebraic curves which are the zero set of polynomials defined over the field of complex numbers.

An interesting question about curves is to find functions on them with prescribed zeroes and poles. The Riemann-Roch theorem finds the dimension of the space of such meromorphic functions. This theorem is a vital tool in the fields of complex analysis and algebraic geometry. It relates the complex analysis of a connected compact Riemann surface with the surface's purely topological property of genus, in a way that can be carried over into purely algebraic settings.

\begin{maintheorem*}[Riemann-Roch]
    Let $D$ be a divisor on a non-singular projective curve $\curveC$ in $\projective_2$ with genus $g$, $\kappa$ be a canonical divisor on $\curveC$.
    $$l(D)-l(\kappa-D)=\deg(D)-g+1$$
\end{maintheorem*}

The Riemann-Roch theorem has many very useful consequences including an easy proof of the law of associativity for the additive group structure on a non-singular cubic curve and a proof that every meromorphic function on a non-singular projective curve is rational.

There are different ways to prove the Riemann–Roch Theorem. One way is to take an analytic approach and study holomorphic and meromorphic functions with Serre duality, which can be found in \cite{ref:miranda}. Another, more modern approach includes the concept of schemes and sheaf cohomology, and an example to this can be found in \cite{ref:hartshorne}. I have utilized resources \cite{ref:kirwan}, \cite{ref:fulton}, \cite{ref:keith}, \cite{ref:hampus}, \cite{ref:terrytao} to study the theorem. This paper uses a more elementary machinery to approach the Riemann-Roch theorem.
